\section{Configuration subsystem}\label{configsys}


In many cases, applications require configuration information to run correctly. For example, a readout program may need to load a set of ADC thresholds in order to manage those ADC's correctly. The application framework provides  for configuration information to be read via Tcl startup scripts.

Configuration data can be directly bound to:\begin{CompactItemize}
\item 
Single C/C++ variables or member data.\item 
Slices of C/C++ arrays.\end{CompactItemize}
In addition, associative arrays (based on the STL Map class) can be created, and initialized via these scripts. The framwoerk provides a wide variety of  configuration policies including:\begin{CompactItemize}
\item 
Read single script.\item 
Read first specified script in directory path list.\item 
Read all specified scripts in a directory path list.\end{CompactItemize}
Once configuration requirments for an application have been determined:\begin{CompactItemize}
\item 
Create a binding for each variable, array slice, or associative array needed for the configuration.\item 
Register these bindings with a CConfiguration\-Manger object.\item 
Request the CCConfiguration manager object process configuration files.\end{CompactItemize}
Configuration scripts can be written using the entire Tcl language. Variable bindings are allowed for data of type:\begin{CompactItemize}
\item 
Integer (int)\item 
Double precision (double)\item 
Boolean (bool)\item 
String (char$\ast$) Each of the configuration classes is templated to support configuration for any of these types. The Configuration bindings classes are:\item 
{\bf CVariable\-Binding} {\rm (p.\,\pageref{classCVariableBinding})} - Allows you to bind a single variable to a Tcl variable in the configuraiton script.\item 
{\bf CArray\-Binding} {\rm (p.\,\pageref{classCArrayBinding})} - Allows you to bind a slice of an array to a Tcl array with integer indices in the configuration script.\item 
{\bf CAssoc\-Array\-Binding} {\rm (p.\,\pageref{classCAssocArrayBinding})} - Allows you to create and configue an associative array with string indices.\end{CompactItemize}
The example below creates all of these bindings, and manages them via a  {\bf CConfiguration\-Manager} {\rm (p.\,\pageref{classCConfigurationManager})} object



\footnotesize\begin{verbatim}//
// Tests the configuration system.
#include <spectrodaq.h>
#include <SpectroFramework.h>
#include <iostream.h>
#include <fstream.h>
#include <vector>
#include <string>
#include <stdlib.h>


class MyApp : public DAQROCNode
{
  // Single configuration variables.

  int    m_nSingle;
  double m_fSingle;
  bool   m_bSingle;
  char*  m_pSingle;

  // Array bindings

  int    m_nArray[100];
  double m_fArray[100];
  bool   m_bArray[100];
  char*  m_pArray[100];

  // pointers to associative array bindings:
  //
  CAssocArrayBinding<int>*    m_pnAssoc;
  CAssocArrayBinding<double>* m_pfAssoc;
  CAssocArrayBinding<bool>*   m_pbAssoc;
  CAssocArrayBinding<char*>*  m_ppAssoc;  
protected:
  int operator()(int argc, char** argv);
  void DumpConfig(CConfigurationManager& rMgr, const char* dumpfile);
};

MyApp theApplication;

int
MyApp::operator()(int argc, char** pargv)
{
  // Bindings for single variables:

  cout << "Binding single variables" << endl;
  CVariableBinding<int>    nSingle(m_nSingle, "Integer", 0);
  CVariableBinding<double> fSingle(m_fSingle,  "Float" , 3.14159);
  CVariableBinding<bool>   bSingle(m_bSingle,  "Bool"  , false);
  CVariableBinding<char*>  pSingle(m_pSingle,  string("Char"), 
                                   (char*)NULL);

  // Bindings for arrays:

  cout << "Binding array slices" << endl;

  CArrayBinding<int>    nArray(m_nArray, 0, 99, "IArray");
  CArrayBinding<double> fArray(m_fArray, 50,99, "FArray");
  CArrayBinding<bool>   bArray(m_bArray, 0, 49, "BArray");
  CArrayBinding<char*>  cArray(m_pArray, 25,75, string("SArray"));

  // Build a bound associative array:

  cout << "Creating associative arrays" << endl;

  CAssocArrayBinding<int>    nAssoc("IAssoc");
  CAssocArrayBinding<double> fAssoc("FAssoc");
  CAssocArrayBinding<bool>   bAssoc("BAssoc");
  CAssocArrayBinding<char*>  cAssoc(string("CAssoc"));
  m_pnAssoc = &nAssoc;
  m_pfAssoc = &fAssoc;
  m_pbAssoc = &bAssoc;
  m_ppAssoc = &cAssoc;

  // Setup the configuration object:

  cout << "Creating configuration manager and loading in bindings:" << endl;
  CConfigurationManager configurer;

  configurer.AddBinding(nSingle);
  configurer.AddBinding(fSingle);
  configurer.AddBinding(bSingle);
  configurer.AddBinding(pSingle);

  list<CTypeFreeBinding*> bindlist;
  bindlist.push_back(&nArray);
  bindlist.push_back(&fArray);
  bindlist.push_back(&bArray);
  bindlist.push_back(&cArray);
  configurer.AddBinding(bindlist);
  configurer.AddBinding(nAssoc);
  configurer.AddBinding(fAssoc);
  configurer.AddBinding(bAssoc);
  configurer.AddBinding(cAssoc);

  //  Configure from a single file:

  cout << "Reading single config file" << endl;

  configurer.ReadConfigFile("config.tcl");
 
  DumpConfig(configurer, "SingleFile");

  // Configure from first file in path list... should get same result.

  cout << "Reading 1st config file in pathlist" << endl;

  vector<string> Pathlist;
  Pathlist.push_back(string("."));
  Pathlist.push_back(string(getenv("HOME")));
  Pathlist.push_back(string(".."));
  configurer.Read1stConfigFile(Pathlist, "config.tcl");
  DumpConfig(configurer, "FirstFile");

  // Configure from multiple files in path list

  cout << "Reading all config files in pathlist." << endl;

  configurer.ReadAllConfigFiles(Pathlist, "config.tcl");
  DumpConfig(configurer, "AllFiles");


}
//

void
MyApp::DumpConfig(CConfigurationManager& rMgr, const char* dumpfile)
{
  // Make the filenames:

  string dumpname(dumpfile);
  string listname(dumpfile);
  dumpname += ".tcl";
  listname += ".txt";
  
  // Do the config dump:

  rMgr.WriteConfigFile(dumpname);

  ofstream txtfile(listname.c_str());
  txtfile << "----------------------- single variables --------------\n";
  txtfile << "m_nSingle = " << m_nSingle;
  txtfile << " m_fSingel = " << m_fSingle;
  txtfile << " m_bSingle = " << m_bSingle;
  txtfile << " m_pSingle = " << (m_pSingle ? m_pSingle : "null") << endl;

  txtfile << "-------------------------- Array variables -------------\n";
  txtfile << "index  int     double   bool     string\n";
  for(int i = 0; i < 100; i++) {
    txtfile << i << m_nArray[i] << " " << m_fArray[i] << " " << m_bArray[i] << " " ;
    txtfile << (m_pArray[i] ? m_pArray[i] : "(null)") << endl;

  }
  txtfile << "----------------------- nAssoc (integer assoc array) ----\n";
  txtfile << "Index           Value\n";
  map<string,int>::iterator i = m_pnAssoc->begin();
  while(i != m_pnAssoc->end()) {
    txtfile << i->first  << " " << i->second << endl;
    i++;
  }
  txtfile << "----------------------fAssoc (double assoc array) ------\n";
  txtfile << "Index           Value\n";
  map<string,double>::iterator f = m_pfAssoc->begin();
  while(f != m_pfAssoc->end()) {
    txtfile << f->first << " " << f->second << endl;
    f++;
  }
  txtfile << "---------------------------bAssoc (Bool assoc array)-----\n";
  txtfile << "Index           Value\n";
  map<string,bool>::iterator b = m_pbAssoc->begin();
  while(b != m_pbAssoc->end()) {
    txtfile << b->first << " " << b->second << endl;
    b++;
  }
  txtfile << "---------------------------cAssoc (char* assoc array) ------\n";
  txtfile << "Index           Value\n";
  map<string,char*>::iterator c = m_ppAssoc->begin();
  while(c != m_ppAssoc->end()) {
    txtfile << c->first << " " << c->second << endl;
    c++;                        
  }


  
  
}
\end{verbatim}\normalsize 


