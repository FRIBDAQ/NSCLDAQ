\section{CLogger  Class Reference}
\label{classCLogger}\index{CLogger@{CLogger}}
{\tt \#include $<$CLogger.h$>$}

\subsection*{Public Types}
\begin{CompactItemize}
\item 
enum {\bf Severity} \{ {\bf SUCCESS}, 
{\bf WARNING}, 
{\bf ERROR}
 \}
\begin{CompactList}\small\item\em The facility name which is logging the event.\item\end{CompactList}\end{CompactItemize}
\subsection*{Public Methods}
\begin{CompactItemize}
\item 
{\bf CLogger} (string facility)
\item 
{\bf CLogger} (const CLogger \&a\-CLogger)
\item 
{\bf $\sim$CLogger} ()
\item 
bool {\bf Log} ({\bf Severity} sev, string message)
\item 
{\bf Host\-List\-Iterator} {\bf begin} ()
\item 
{\bf Host\-List\-Iterator} {\bf end} ()
\item 
int {\bf size} ()
\item 
void {\bf Add\-Host} (const string \&new\-Host)
\item 
void {\bf Remove\-Host} (const string \&old\-Host)
\item 
void {\bf Remove\-Host} ({\bf Host\-List\-Iterator} It)
\end{CompactItemize}
\subsection*{Private Methods}
\begin{CompactItemize}
\item 
CLogger {\bf operator=} (const CLogger \&a\-CLogger)
\item 
int {\bf operator==} (const CLogger \&a\-CLogger)
\item 
list$<$ string $>$ {\bf get\-Host\-List} () const
\item 
const string {\bf get\-Facility} () const
\end{CompactItemize}
\subsection*{Private Attributes}
\begin{CompactItemize}
\item 
list$<$ string $>$ {\bf m\_\-Host\-List}
\item 
const string {\bf m\_\-s\-Facility}
\begin{CompactList}\small\item\em The list of hosts running Event\-Log.tcl.\item\end{CompactList}\end{CompactItemize}


\subsection{Detailed Description}
Encapsulates a logger class for events which occur in the data acquisition framework.  The Logger maintains a list of hostnames on which the actual Event\-Log.tcl script is running, as well as the port number and name of the facility which contains it. An event is logged via a call to the {\bf CLogger::Log}() {\rm (p.\,\pageref{classCLogger_a3})} function member which creates  client sockets that connect to each of the hosts in the host list and logs the events. Each socket is shut down immediately  after the log entry is written. CLogger also contains an enumeration {\bf CLogger::Severity} {\rm (p.\,\pageref{classCLogger_s3})}. This is used for function {\bf CLogger::Log}() {\rm (p.\,\pageref{classCLogger_a3})}, as there are only three possible values for severity that the logger will accept:

\begin{CompactItemize}
\item 
SUCCESS: The operation being logged completed successfully.\item 
WARNING: The operation being logged completed, but there was a warning issued by the application.\item 
ERROR: The operation being logged did not complete, and an exceptional condition was encountered. \end{CompactItemize}




Definition at line 347 of file CLogger.h.

\subsection{Member Enumeration Documentation}
\index{CLogger@{CLogger}!Severity@{Severity}}
\index{Severity@{Severity}!CLogger@{CLogger}}
\subsubsection{\setlength{\rightskip}{0pt plus 5cm}enum CLogger::Severity}\label{classCLogger_s3}


The facility name which is logging the event.

\begin{Desc}
\item[Enumeration values:]\par
\begin{description}
\index{SUCCESS@{SUCCESS}!CLogger@{CLogger}}\index{CLogger@{CLogger}!SUCCESS@{SUCCESS}}\item[{\em 
{\em SUCCESS}\label{classCLogger_s3s0}
}]\index{WARNING@{WARNING}!CLogger@{CLogger}}\index{CLogger@{CLogger}!WARNING@{WARNING}}\item[{\em 
{\em WARNING}\label{classCLogger_s3s1}
}]\index{ERROR@{ERROR}!CLogger@{CLogger}}\index{CLogger@{CLogger}!ERROR@{ERROR}}\item[{\em 
{\em ERROR}\label{classCLogger_s3s2}
}]\end{description}
\end{Desc}



Definition at line 353 of file CLogger.h.

\subsection{Constructor \& Destructor Documentation}
\index{CLogger@{CLogger}!CLogger@{CLogger}}
\index{CLogger@{CLogger}!CLogger@{CLogger}}
\subsubsection{\setlength{\rightskip}{0pt plus 5cm}CLogger::CLogger (string {\em facility})}\label{classCLogger_a0}


\char`\"{}Default Constructor\char`\"{} This is the default constructor which constructs a CLogger given a list of hosts to which it will form socket connections when logging events, and a facility name which will be the name of the facility doing the logging.\begin{Desc}
\item[Parameters: ]\par
\begin{description}
\item[{\em 
facility}]- the name of the facility doing the logging \end{description}
\end{Desc}


Definition at line 310 of file CLogger.cpp.\index{CLogger@{CLogger}!CLogger@{CLogger}}
\index{CLogger@{CLogger}!CLogger@{CLogger}}
\subsubsection{\setlength{\rightskip}{0pt plus 5cm}CLogger::CLogger (const CLogger \& {\em a\-CLogger})}\label{classCLogger_a1}


\char`\"{}Copy Constructor\char`\"{} This is the copy constructor. It creates a new object by copying the information of the reference object which is its parameter.\begin{Desc}
\item[Parameters: ]\par
\begin{description}
\item[{\em 
a\-CLogger}]- the reference object whose attributes will be copied. \end{description}
\end{Desc}


Definition at line 320 of file CLogger.cpp.\index{CLogger@{CLogger}!~CLogger@{$\sim$CLogger}}
\index{~CLogger@{$\sim$CLogger}!CLogger@{CLogger}}
\subsubsection{\setlength{\rightskip}{0pt plus 5cm}CLogger::$\sim$CLogger ()}\label{classCLogger_a2}


\char`\"{}Destructor\char`\"{} Called when an object goes out of scope, or when execution of the program is terminated. Destroys the object, and frees up space. 

Definition at line 329 of file CLogger.cpp.

\subsection{Member Function Documentation}
\index{CLogger@{CLogger}!AddHost@{AddHost}}
\index{AddHost@{AddHost}!CLogger@{CLogger}}
\subsubsection{\setlength{\rightskip}{0pt plus 5cm}void CLogger::Add\-Host (const string \& {\em new\-Host})}\label{classCLogger_a7}


Operation Type: Mutator

Purpose: Adds a host to the list of hosts that we will attempt to form a connection with and log a message to.\begin{Desc}
\item[Parameters: ]\par
\begin{description}
\item[{\em 
const}]string\& new\-Host The name of the new host to be  added to m\_\-Host\-List\end{description}
\end{Desc}
\begin{Desc}
\item[Exceptions: ]\par
\begin{description}
\item[{\em 
{\bf CDuplicate\-Name\-Exception} {\rm (p.\,\pageref{classCDuplicateNameException})}}] Thrown if the host already  exists in the host list \end{description}
\end{Desc}


Definition at line 517 of file CLogger.cpp.

References Host\-List\-Iterator, and m\_\-Host\-List.\index{CLogger@{CLogger}!begin@{begin}}
\index{begin@{begin}!CLogger@{CLogger}}
\subsubsection{\setlength{\rightskip}{0pt plus 5cm}{\bf Host\-List\-Iterator} CLogger::begin ()}\label{classCLogger_a4}


Operation Type: Selector

Purpose: Returns an iterator to the first host in m\_\-Host\-List. 

Definition at line 468 of file CLogger.cpp.

References m\_\-Host\-List.\index{CLogger@{CLogger}!end@{end}}
\index{end@{end}!CLogger@{CLogger}}
\subsubsection{\setlength{\rightskip}{0pt plus 5cm}{\bf Host\-List\-Iterator} CLogger::end ()}\label{classCLogger_a5}


Operator Type: Selector

Purpose: Returns an iterator which points to just past the last host in m\_\-Host\-List. 

Definition at line 482 of file CLogger.cpp.

References m\_\-Host\-List.\index{CLogger@{CLogger}!getFacility@{getFacility}}
\index{getFacility@{getFacility}!CLogger@{CLogger}}
\subsubsection{\setlength{\rightskip}{0pt plus 5cm}const string CLogger::get\-Facility () const\hspace{0.3cm}{\tt  [inline, private]}}\label{classCLogger_c3}




Definition at line 377 of file CLogger.h.

References m\_\-s\-Facility.\index{CLogger@{CLogger}!getHostList@{getHostList}}
\index{getHostList@{getHostList}!CLogger@{CLogger}}
\subsubsection{\setlength{\rightskip}{0pt plus 5cm}list$<$string$>$ CLogger::get\-Host\-List () const\hspace{0.3cm}{\tt  [inline, private]}}\label{classCLogger_c2}




Definition at line 376 of file CLogger.h.

References m\_\-Host\-List.\index{CLogger@{CLogger}!Log@{Log}}
\index{Log@{Log}!CLogger@{CLogger}}
\subsubsection{\setlength{\rightskip}{0pt plus 5cm}bool CLogger::Log ({\bf Severity} {\em sev}, string {\em message})}\label{classCLogger_a3}


Operation Type: Log to file and screen

Attempts to log a message (facility, severity, message, date) to  Event\-Log.tcl by opening a socket connection to each of the hosts in m\_\-Host\-List. The first connection logs the message which consists of the facility, severity, and message to the log file via a call to the tcl procedure Logger::Log. This only needs to be written once, since the same log file will be used by each displayer. The Logger::Log tcl procedure obtains the exact time of the event via a call to date(1). It then returns the date string through the socket (which we read) and that string is what is displayed on the GUI via a call to Logger::Display\_\-Event. In this way, the date on each display will be exactly the same. NOTE: If {\bf Log}() {\rm (p.\,\pageref{classCLogger_a3})} fails for the first host in the list, then the event will not be written to the log file and will not be displayed anywhere.  If connection to the first host succeeds, but fails on a subsequent host, the event will be logged to file but will not be displayed on whichever host connection failed. It will, however, be displayed on all hosts (including subsequent hosts) to which connection does not fail.\begin{Desc}
\item[Exceptions: ]\par
\begin{description}
\item[{\em 
{\bf CErrno\-Exception} {\rm (p.\,\pageref{classCErrnoException})}}] - Errno exception occurred \item[{\em 
{\bf CTCPBad\-Socket\-State} {\rm (p.\,\pageref{classCTCPBadSocketState})}}] - {\bf CSocket::m\_\-State} {\rm (p.\,\pageref{classCSocket_o1})} was not disconnected \item[{\em 
{\bf CTCPNo\-Such\-Host} {\rm (p.\,\pageref{classCTCPNoSuchHost})}}] - Host not in DNS or nonexistent \item[{\em 
{\bf CTCPNo\-Such\-Service} {\rm (p.\,\pageref{classCTCPNoSuchService})}}] - Named service does not translate. \item[{\em 
{\bf CTCPConnection\-Failed} {\rm (p.\,\pageref{classCTCPConnectionFailed})}}] - Connection refused by remote host \item[{\em 
{\bf CTCPConnection\-Lost} {\rm (p.\,\pageref{classCTCPConnectionLost})}}] - Connection terminated by remote host\end{description}
\end{Desc}
\begin{Desc}
\item[Parameters: ]\par
\begin{description}
\item[{\em 
sev}]- This is an enumerated value which represent the severity of the event. \item[{\em 
message}]- This is the message which the caller wants to log. \end{description}
\end{Desc}


Definition at line 365 of file CLogger.cpp.

References CSocket::Connect(), ERROR, m\_\-Host\-List, m\_\-s\-Facility, PORT, CSocket::Read(), CSocket::Shutdown(), SUCCESS, WARNING, and CSocket::Write().\index{CLogger@{CLogger}!operator=@{operator=}}
\index{operator=@{operator=}!CLogger@{CLogger}}
\subsubsection{\setlength{\rightskip}{0pt plus 5cm}CLogger CLogger::operator= (const CLogger \& {\em a\-CLogger})\hspace{0.3cm}{\tt  [private]}}\label{classCLogger_c0}


\index{CLogger@{CLogger}!operator==@{operator==}}
\index{operator==@{operator==}!CLogger@{CLogger}}
\subsubsection{\setlength{\rightskip}{0pt plus 5cm}int CLogger::operator== (const CLogger \& {\em a\-CLogger})\hspace{0.3cm}{\tt  [private]}}\label{classCLogger_c1}


\index{CLogger@{CLogger}!RemoveHost@{RemoveHost}}
\index{RemoveHost@{RemoveHost}!CLogger@{CLogger}}
\subsubsection{\setlength{\rightskip}{0pt plus 5cm}void CLogger::Remove\-Host ({\bf Host\-List\-Iterator} {\em It})}\label{classCLogger_a9}


Operation Type: Mutator

Purpose: Removes a host from the list of hosts that we will attempt to form a connection with and log a message to.\begin{Desc}
\item[Parameters: ]\par
\begin{description}
\item[{\em 
Host\-List\-Iterator}]It An iterator which points to the host in m\_\-Host\-List that is to be removed. \end{description}
\end{Desc}
\begin{Desc}
\item[Exceptions: ]\par
\begin{description}
\item[{\em 
{\bf CNo\-Such\-Object\-Exception} {\rm (p.\,\pageref{classCNoSuchObjectException})}}] Thrown if the name supplied in the paramter is not in m\_\-Host\-List. \end{description}
\end{Desc}


Definition at line 574 of file CLogger.cpp.

References Host\-List\-Iterator, and m\_\-Host\-List.\index{CLogger@{CLogger}!RemoveHost@{RemoveHost}}
\index{RemoveHost@{RemoveHost}!CLogger@{CLogger}}
\subsubsection{\setlength{\rightskip}{0pt plus 5cm}void CLogger::Remove\-Host (const string \& {\em old\-Host})}\label{classCLogger_a8}


Operation Type: Mutator

Purpose: Removes a host from the list of hosts that we will attempt to form a connection with and log a message to.\begin{Desc}
\item[Parameters: ]\par
\begin{description}
\item[{\em 
const}]string\& old\-Host The name of the old host to be  removed from m\_\-Host\-List \end{description}
\end{Desc}
\begin{Desc}
\item[Exceptions: ]\par
\begin{description}
\item[{\em 
{\bf CNo\-Such\-Object\-Exception} {\rm (p.\,\pageref{classCNoSuchObjectException})}}] Thrown if the name supplied in the parameter is not in m\_\-Host\-List. \end{description}
\end{Desc}


Definition at line 544 of file CLogger.cpp.

References Host\-List\-Iterator, and m\_\-Host\-List.\index{CLogger@{CLogger}!size@{size}}
\index{size@{size}!CLogger@{CLogger}}
\subsubsection{\setlength{\rightskip}{0pt plus 5cm}int CLogger::size ()}\label{classCLogger_a6}


Operation Type: Selector

Purpose: Returns the number of hosts that are currently being logged to.

\begin{Desc}
\item[Returns: ]\par
The number of hosts in m\_\-Host\-List \end{Desc}


Definition at line 497 of file CLogger.cpp.

References m\_\-Host\-List.

\subsection{Member Data Documentation}
\index{CLogger@{CLogger}!m_HostList@{m\_\-HostList}}
\index{m_HostList@{m\_\-HostList}!CLogger@{CLogger}}
\subsubsection{\setlength{\rightskip}{0pt plus 5cm}list$<$string$>$ CLogger::m\_\-Host\-List\hspace{0.3cm}{\tt  [private]}}\label{classCLogger_o0}




Definition at line 349 of file CLogger.h.

Referenced by Add\-Host(), begin(), end(), get\-Host\-List(), Log(), Remove\-Host(), and size().\index{CLogger@{CLogger}!m_sFacility@{m\_\-sFacility}}
\index{m_sFacility@{m\_\-sFacility}!CLogger@{CLogger}}
\subsubsection{\setlength{\rightskip}{0pt plus 5cm}const string CLogger::m\_\-s\-Facility\hspace{0.3cm}{\tt  [private]}}\label{classCLogger_o1}


The list of hosts running Event\-Log.tcl.



Definition at line 350 of file CLogger.h.

Referenced by get\-Facility(), and Log().

The documentation for this class was generated from the following files:\begin{CompactItemize}
\item 
{\bf CLogger.h}\item 
{\bf CLogger.cpp}\end{CompactItemize}
