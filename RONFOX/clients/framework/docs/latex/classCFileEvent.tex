\section{CFile\-Event  Class Reference}
\label{classCFileEvent}\index{CFileEvent@{CFile\-Event}}
{\tt \#include $<$CFile\-Event.h$>$}

Inheritance diagram for CFile\-Event::\begin{figure}[H]
\begin{center}
\leavevmode
\includegraphics[height=5cm]{classCFileEvent}
\end{center}
\end{figure}
\subsection*{Public Methods}
\begin{CompactItemize}
\item 
{\bf CFile\-Event} (int fd, int access={\bf readable})
\item 
{\bf CFile\-Event} (const char $\ast$p\-Name, int flags=O\_\-RDONLY, int access={\bf readable})
\item 
{\bf CFile\-Event} (const string \&r\-Name, int flags=O\_\-RDONLY, int access={\bf readable})
\item 
{\bf CFile\-Event} (int fd, const char $\ast$p\-Obj\-Name, int access={\bf readable})
\item 
{\bf CFile\-Event} (const char $\ast$p\-Obj\-Name, const char $\ast$p\-Name, int flags=O\_\-RDONLY, int access={\bf readable})
\item 
{\bf CFile\-Event} (const char $\ast$p\-Obj\-Name, const string \&r\-Name, int flags=O\_\-RDONLY, int access={\bf readable})
\item 
{\bf CFile\-Event} (int fd, const string \&r\-Obj\-Name, int access={\bf readable})
\item 
{\bf CFile\-Event} (const string \&r\-Obj\-Name, const char $\ast$pname, int flags=O\_\-RDONLY, int access={\bf readable})
\item 
{\bf CFile\-Event} (const string \&r\-Obj\-Name, const string \&r\-Name, int flags=O\_\-RDONLY, int access={\bf readable})
\item 
{\bf $\sim$CFile\-Event} ()
\item 
int {\bf get\-Fd} () const
\item 
int {\bf get\-Close\-On\-Destory} () const
\item 
void {\bf Monitor\-Readable} (bool f\-Readable=true)
\item 
void {\bf Monitor\-Writable} (bool f\-Writable=true)
\item 
void {\bf Monitor\-Exceptions} (bool f\-Except=true)
\item 
virtual void {\bf On\-Readable} (istream \&r\-Stream)
\item 
virtual void {\bf On\-Writable} (ostream \&r\-Stream)
\item 
virtual void {\bf On\-Exception} (iostream \&fd)
\item 
virtual void {\bf On\-Timeout} (iostream \&str)
\item 
virtual string {\bf Describe\-Self} ()
\end{CompactItemize}
\subsection*{Static Public Attributes}
\begin{CompactItemize}
\item 
int {\bf readable} = 1
\begin{CompactList}\small\item\em Bitmask for readable monitoring.\item\end{CompactList}\item 
int {\bf writeable} = 2
\begin{CompactList}\small\item\em Bitmask for writable monitoring.\item\end{CompactList}\item 
int {\bf exceptions} = 4
\begin{CompactList}\small\item\em Bitmask for exception monitoring.\item\end{CompactList}\end{CompactItemize}
\subsection*{Protected Methods}
\begin{CompactItemize}
\item 
void {\bf Exit} (int status)
\item 
void {\bf Setup\-Monitor} (int Access\-Mask)
\item 
{\bf CFd\-Monitor} $\ast$ {\bf Create\-Monitor} (const char $\ast$p\-Filename, const char $\ast$p\-Monitor, int flags)
\end{CompactItemize}
\subsection*{Private Methods}
\begin{CompactItemize}
\item 
{\bf CFile\-Event} (const CFile\-Event \&r\-Event)
\item 
CFile\-Event \& {\bf operator=} (const CFile\-Event \&rhs)
\item 
int {\bf operator==} (const CFile\-Event \&r\-Event)
\end{CompactItemize}
\subsection*{Private Attributes}
\begin{CompactItemize}
\item 
int {\bf m\_\-n\-Fd}
\begin{CompactList}\small\item\em File descriptor.\item\end{CompactList}\item 
bool {\bf m\_\-f\-Close\-On\-Destroy}
\begin{CompactList}\small\item\em true if we opened the file.\item\end{CompactList}\end{CompactItemize}


\subsection{Detailed Description}
C lass providing functionality for  file descriptor events. Must be derived and  operator() implemented to provide application functionality. 



Definition at line 333 of file CFile\-Event.h.

\subsection{Constructor \& Destructor Documentation}
\index{CFileEvent@{CFile\-Event}!CFileEvent@{CFileEvent}}
\index{CFileEvent@{CFileEvent}!CFileEvent@{CFile\-Event}}
\subsubsection{\setlength{\rightskip}{0pt plus 5cm}CFile\-Event::CFile\-Event (int {\em fd}, int {\em access} = {\bf readable})}\label{classCFileEvent_a0}


Construct an anonymous file event from a file descriptor,  The file monitor is a {\bf CFd\-Monitor} {\rm (p.\,\pageref{classCFdMonitor})} directly constructed from the fd via new. \begin{Desc}
\item[Parameters: ]\par
\begin{description}
\item[{\em 
fd}]- File descriptor to open on the file. \item[{\em 
access}]- Bitwise or of access required to file. \end{description}
\end{Desc}


Definition at line 404 of file CFile\-Event.cpp.

References CNamed\-Object::Append\-Class\-Info(), CNamed\-Object::Get\-Auto\-Name(), m\_\-f\-Close\-On\-Destroy, m\_\-n\-Fd, and Setup\-Monitor().\index{CFileEvent@{CFile\-Event}!CFileEvent@{CFileEvent}}
\index{CFileEvent@{CFileEvent}!CFileEvent@{CFile\-Event}}
\subsubsection{\setlength{\rightskip}{0pt plus 5cm}CFile\-Event::CFile\-Event (const char $\ast$ {\em p\-Name}, int {\em flags} = O\_\-RDONLY, int {\em access} = {\bf readable})}\label{classCFileEvent_a1}


Construct an anonymous file event from a filename as a const char$\ast$ A file is opened and created with Create\-Monitor. \begin{Desc}
\item[Parameters: ]\par
\begin{description}
\item[{\em 
p\-Name}]- Name of file to open/create. \item[{\em 
flags}]- open(2) flags. \item[{\em 
access}]- Bitwise access requirements for the file. \end{description}
\end{Desc}


Definition at line 421 of file CFile\-Event.cpp.

References CNamed\-Object::Append\-Class\-Info(), and Setup\-Monitor().\index{CFileEvent@{CFile\-Event}!CFileEvent@{CFileEvent}}
\index{CFileEvent@{CFileEvent}!CFileEvent@{CFile\-Event}}
\subsubsection{\setlength{\rightskip}{0pt plus 5cm}CFile\-Event::CFile\-Event (const string \& {\em r\-Name}, int {\em flags} = O\_\-RDONLY, int {\em access} = {\bf readable})}\label{classCFileEvent_a2}


Construct an anonymous file event from a filename as a const string\& A file is opened and created with {\bf Create\-Monitor}() {\rm (p.\,\pageref{classCFileEvent_b2})} \begin{Desc}
\item[Parameters: ]\par
\begin{description}
\item[{\em 
r\-Name}]- Nameof the file to open/create \item[{\em 
flags}]- open(2) flags.. \item[{\em 
access}]- Bitwise access requirements for the file. \end{description}
\end{Desc}


Definition at line 436 of file CFile\-Event.cpp.

References CNamed\-Object::Append\-Class\-Info(), and Setup\-Monitor().\index{CFileEvent@{CFile\-Event}!CFileEvent@{CFileEvent}}
\index{CFileEvent@{CFileEvent}!CFileEvent@{CFile\-Event}}
\subsubsection{\setlength{\rightskip}{0pt plus 5cm}CFile\-Event::CFile\-Event (int {\em fd}, const char $\ast$ {\em p\-Obj\-Name}, int {\em access} = {\bf readable})}\label{classCFileEvent_a3}


Construct a named file event from a char$\ast$ object name, and a file descriptor. The {\bf CFd\-Monitor} {\rm (p.\,\pageref{classCFdMonitor})} is directly new'd into existence in the operation. \begin{Desc}
\item[Parameters: ]\par
\begin{description}
\item[{\em 
p\-Obj\-Name}]- Pointer to object name. \item[{\em 
fd}]- File descriptor on already open file to monitor. \item[{\em 
access}]- Bitwise or of the desired access to the file.\end{description}
\end{Desc}
Throws:\begin{CompactItemize}
\item 
{\bf CDuplicate\-Name\-Exception} {\rm (p.\,\pageref{classCDuplicateNameException})} - if this object or any subobject already have this name. \end{CompactItemize}


Definition at line 463 of file CFile\-Event.cpp.

References CNamed\-Object::Append\-Class\-Info(), and Setup\-Monitor().\index{CFileEvent@{CFile\-Event}!CFileEvent@{CFileEvent}}
\index{CFileEvent@{CFileEvent}!CFileEvent@{CFile\-Event}}
\subsubsection{\setlength{\rightskip}{0pt plus 5cm}CFile\-Event::CFile\-Event (const char $\ast$ {\em p\-Obj\-Name}, const char $\ast$ {\em p\-Filename}, int {\em flags} = O\_\-RDONLY, int {\em access} = {\bf readable})}\label{classCFileEvent_a4}


Construct a named file event from a char$\ast$ object name and a char$\ast$ filename. The {\bf CFd\-Monitor} {\rm (p.\,\pageref{classCFdMonitor})} is created by Create\-Monitor. \begin{Desc}
\item[Parameters: ]\par
\begin{description}
\item[{\em 
p\-Obj\-Name}]- Name to give to all the objects. \item[{\em 
p\-Filename}]- Name of the file to access. \item[{\em 
flags}]- open(2) flags. \item[{\em 
access}]- Bitwise or of the access rights desired to the file.\end{description}
\end{Desc}
Throws:\begin{CompactItemize}
\item 
{\bf CDuplicate\-Name\-Exception} {\rm (p.\,\pageref{classCDuplicateNameException})} - if this object or any subobject already have this name. \end{CompactItemize}


Definition at line 489 of file CFile\-Event.cpp.

References CNamed\-Object::Append\-Class\-Info(), and Setup\-Monitor().\index{CFileEvent@{CFile\-Event}!CFileEvent@{CFileEvent}}
\index{CFileEvent@{CFileEvent}!CFileEvent@{CFile\-Event}}
\subsubsection{\setlength{\rightskip}{0pt plus 5cm}CFile\-Event::CFile\-Event (const char $\ast$ {\em p\-Obj\-Name}, const string \& {\em r\-Filename}, int {\em flags} = O\_\-RDONLY, int {\em access} = {\bf readable})}\label{classCFileEvent_a5}


Constructs a named file event given a char$\ast$ object name and a string\& filename. The monitor is created via Create\-Monitor. \begin{Desc}
\item[Parameters: ]\par
\begin{description}
\item[{\em 
p\-Obj\-Name}]- Pointer to the name of the object  \item[{\em 
r\-Filename}]- Reference to the name of the file \item[{\em 
flags}]- open(2) flags.. \item[{\em 
access}]- Bitmask specifying access desired.\end{description}
\end{Desc}
Throws:\begin{CompactItemize}
\item 
{\bf CDuplicate\-Name\-Exception} {\rm (p.\,\pageref{classCDuplicateNameException})} - if this object or any subobject already have this name. \end{CompactItemize}


Definition at line 513 of file CFile\-Event.cpp.

References CNamed\-Object::Append\-Class\-Info(), and Setup\-Monitor().\index{CFileEvent@{CFile\-Event}!CFileEvent@{CFileEvent}}
\index{CFileEvent@{CFileEvent}!CFileEvent@{CFile\-Event}}
\subsubsection{\setlength{\rightskip}{0pt plus 5cm}CFile\-Event::CFile\-Event (int {\em fd}, const string \& {\em r\-Obj\-Name}, int {\em access} = {\bf readable})}\label{classCFileEvent_a6}


Construct a file event from a string\& object name and an fd. The Monitor can be created directly via new \begin{Desc}
\item[Parameters: ]\par
\begin{description}
\item[{\em 
r\-Obj\-Name}]- Reference to the name of the object. \item[{\em 
fd}]- File descriptor to monitor. \item[{\em 
access}]- Desired access to the file.\end{description}
\end{Desc}
Throws:\begin{CompactItemize}
\item 
{\bf CDuplicate\-Name\-Exception} {\rm (p.\,\pageref{classCDuplicateNameException})} - if this object or any subobject already have this name. \end{CompactItemize}


Definition at line 539 of file CFile\-Event.cpp.

References CNamed\-Object::Append\-Class\-Info(), and Setup\-Monitor().\index{CFileEvent@{CFile\-Event}!CFileEvent@{CFileEvent}}
\index{CFileEvent@{CFileEvent}!CFileEvent@{CFile\-Event}}
\subsubsection{\setlength{\rightskip}{0pt plus 5cm}CFile\-Event::CFile\-Event (const string \& {\em r\-Obj\-Name}, const char $\ast$ {\em p\-Filename}, int {\em flags} = O\_\-RDONLY, int {\em access} = {\bf readable})}\label{classCFileEvent_a7}


Construct a file event from a string\& object name and a char$\ast$ filename. The monitor is created via Create\-Monitor. \begin{Desc}
\item[Parameters: ]\par
\begin{description}
\item[{\em 
r\-Obj\-Name}]- Reference to the name of the object. \item[{\em 
p\-Filename}]- Pointer to the filename. \item[{\em 
flags}]- open(2) flags. \item[{\em 
access}]- Bitmask specifying the desired file access.\end{description}
\end{Desc}
Throws:\begin{CompactItemize}
\item 
{\bf CDuplicate\-Name\-Exception} {\rm (p.\,\pageref{classCDuplicateNameException})} - if this object or any subobject already have this name. \end{CompactItemize}


Definition at line 562 of file CFile\-Event.cpp.

References CNamed\-Object::Append\-Class\-Info(), and Setup\-Monitor().\index{CFileEvent@{CFile\-Event}!CFileEvent@{CFileEvent}}
\index{CFileEvent@{CFileEvent}!CFileEvent@{CFile\-Event}}
\subsubsection{\setlength{\rightskip}{0pt plus 5cm}CFile\-Event::CFile\-Event (const string \& {\em r\-Obj\-Name}, const string \& {\em r\-Filename}, int {\em flags} = O\_\-RDONLY, int {\em access} = {\bf readable})}\label{classCFileEvent_a8}


Construct a file event from a string\& object name, and a string\& filename. The monitor is created via Create\-Monitor: \begin{Desc}
\item[Parameters: ]\par
\begin{description}
\item[{\em 
r\-Obj\-Name}]- The name of the object being created. \item[{\em 
r\-Filename}]- The name of the file being accessed. \item[{\em 
flags}]- open(2) flags. \item[{\em 
access}]- bitmask of requested access.\end{description}
\end{Desc}
Throws:\begin{CompactItemize}
\item 
{\bf CDuplicate\-Name\-Exception} {\rm (p.\,\pageref{classCDuplicateNameException})} - if this object or any subobject already have this name. \end{CompactItemize}


Definition at line 584 of file CFile\-Event.cpp.

References CNamed\-Object::Append\-Class\-Info(), and Setup\-Monitor().\index{CFileEvent@{CFile\-Event}!~CFileEvent@{$\sim$CFileEvent}}
\index{~CFileEvent@{$\sim$CFileEvent}!CFileEvent@{CFile\-Event}}
\subsubsection{\setlength{\rightskip}{0pt plus 5cm}CFile\-Event::$\sim$CFile\-Event ()}\label{classCFileEvent_a9}


Destructor: In all of the construction methods, the monitor and reactor are dynamically instantiated via new (that's what Create\-Monitor will do). We need to get the Monitor and the Reactor and delete them. 

Definition at line 600 of file CFile\-Event.cpp.

References CEvent::get\-Monitor(), CEvent::get\-Reactor(), and m\_\-n\-Fd.\index{CFileEvent@{CFile\-Event}!CFileEvent@{CFileEvent}}
\index{CFileEvent@{CFileEvent}!CFileEvent@{CFile\-Event}}
\subsubsection{\setlength{\rightskip}{0pt plus 5cm}CFile\-Event::CFile\-Event (const CFile\-Event \& {\em r\-Event})\hspace{0.3cm}{\tt  [private]}}\label{classCFileEvent_c0}




\subsection{Member Function Documentation}
\index{CFileEvent@{CFile\-Event}!CreateMonitor@{CreateMonitor}}
\index{CreateMonitor@{CreateMonitor}!CFileEvent@{CFile\-Event}}
\subsubsection{\setlength{\rightskip}{0pt plus 5cm}{\bf CFd\-Monitor} $\ast$ CFile\-Event::Create\-Monitor (const char $\ast$ {\em p\-Filename}, const char $\ast$ {\em p\-Monitor}, int {\em flags})\hspace{0.3cm}{\tt  [protected]}}\label{classCFileEvent_b2}


Creates a new monitor given a filename. The file is created/opened for  read/write and the fd resulting is used to create a new {\bf CFd\-Monitor} {\rm (p.\,\pageref{classCFdMonitor})} (via new).\begin{Desc}
\item[Parameters: ]\par
\begin{description}
\item[{\em 
p\-Filename}]- Name of the file to create/open. \item[{\em 
p\-Monitor}]- Name to give to the monitor. \item[{\em 
flags}]- open(2) flags.\end{description}
\end{Desc}
Exceptions:\begin{CompactItemize}
\item 
{\bf CErrno\-Exception} {\rm (p.\,\pageref{classCErrnoException})} - If the file cannot be created/opened.\item 
{\bf CDuplicate\-Name\-Exception} {\rm (p.\,\pageref{classCDuplicateNameException})} - If the a monitor with that name already exists. \end{CompactItemize}


Definition at line 728 of file CFile\-Event.cpp.

References m\_\-n\-Fd.\index{CFileEvent@{CFile\-Event}!DescribeSelf@{DescribeSelf}}
\index{DescribeSelf@{DescribeSelf}!CFileEvent@{CFile\-Event}}
\subsubsection{\setlength{\rightskip}{0pt plus 5cm}string CFile\-Event::Describe\-Self ()\hspace{0.3cm}{\tt  [virtual]}}\label{classCFileEvent_a19}


Called to get a description of the object. Tells the world we're a file handling event, dumps the {\bf CEvent} {\rm (p.\,\pageref{classCEvent})} Description,  and gives the values of m\_\-n\-Fd, m\_\-f\-Close\-On\-Destroy: 

Reimplemented from {\bf CEvent} {\rm (p.\,\pageref{classCEvent_a16})}.

Reimplemented in {\bf CServer\-Connection\-Event} {\rm (p.\,\pageref{classCServerConnectionEvent_a14})}, and {\bf CServer\-Instance} {\rm (p.\,\pageref{classCServerInstance_a11})}.

Definition at line 785 of file CFile\-Event.cpp.

References CEvent::Describe\-Self(), m\_\-f\-Close\-On\-Destroy, and m\_\-n\-Fd.\index{CFileEvent@{CFile\-Event}!Exit@{Exit}}
\index{Exit@{Exit}!CFileEvent@{CFile\-Event}}
\subsubsection{\setlength{\rightskip}{0pt plus 5cm}void CFile\-Event::Exit (int {\em status})\hspace{0.3cm}{\tt  [protected]}}\label{classCFileEvent_b0}


Exits the thread now (the object must still be destroyed at some point). The global mutex is released unconditionally, a DAQStatus object is created with the given status value and DAQThread::Exit is called.\begin{Desc}
\item[Parameters: ]\par
\begin{description}
\item[{\em 
status}]- An integer status value (often chosen from errno.h). \end{description}
\end{Desc}


Definition at line 706 of file CFile\-Event.cpp.

References CApplication\-Serializer::get\-Instance(), and CThread\-Recursive\-Mutex::Un\-Lock\-Completely().\index{CFileEvent@{CFile\-Event}!getCloseOnDestory@{getCloseOnDestory}}
\index{getCloseOnDestory@{getCloseOnDestory}!CFileEvent@{CFile\-Event}}
\subsubsection{\setlength{\rightskip}{0pt plus 5cm}int CFile\-Event::get\-Close\-On\-Destory () const\hspace{0.3cm}{\tt  [inline]}}\label{classCFileEvent_a11}




Definition at line 405 of file CFile\-Event.h.

References m\_\-f\-Close\-On\-Destroy.\index{CFileEvent@{CFile\-Event}!getFd@{getFd}}
\index{getFd@{getFd}!CFileEvent@{CFile\-Event}}
\subsubsection{\setlength{\rightskip}{0pt plus 5cm}int CFile\-Event::get\-Fd () const\hspace{0.3cm}{\tt  [inline]}}\label{classCFileEvent_a10}




Definition at line 404 of file CFile\-Event.h.

References m\_\-n\-Fd.

Referenced by CFile\-Event\-Reactor::On\-Timeout().\index{CFileEvent@{CFile\-Event}!MonitorExceptions@{MonitorExceptions}}
\index{MonitorExceptions@{MonitorExceptions}!CFileEvent@{CFile\-Event}}
\subsubsection{\setlength{\rightskip}{0pt plus 5cm}void CFile\-Event::Monitor\-Exceptions (bool {\em f\-Except} = true)}\label{classCFileEvent_a14}


Select if the monitor should watch exceptions on the fd. This is delegated to the monitor's Monitor\-Exceptions member function. \begin{Desc}
\item[Parameters: ]\par
\begin{description}
\item[{\em 
f\-Except}]- true to watch for exceptions, false to disable. \end{description}
\end{Desc}


Definition at line 636 of file CFile\-Event.cpp.

References CEvent::get\-Monitor().

Referenced by Setup\-Monitor().\index{CFileEvent@{CFile\-Event}!MonitorReadable@{MonitorReadable}}
\index{MonitorReadable@{MonitorReadable}!CFileEvent@{CFile\-Event}}
\subsubsection{\setlength{\rightskip}{0pt plus 5cm}void CFile\-Event::Monitor\-Readable (bool {\em f\-Readable} = true)}\label{classCFileEvent_a12}


Select if the monitor should watch readability. This is delegated to the monitor's Monitor\-Readable member function. \begin{Desc}
\item[Parameters: ]\par
\begin{description}
\item[{\em 
f\-Readable}]- true to watch readability false to disable that. \end{description}
\end{Desc}


Definition at line 616 of file CFile\-Event.cpp.

References CEvent::get\-Monitor().

Referenced by Setup\-Monitor(), and CServer\-Instance::Shutdown().\index{CFileEvent@{CFile\-Event}!MonitorWritable@{MonitorWritable}}
\index{MonitorWritable@{MonitorWritable}!CFileEvent@{CFile\-Event}}
\subsubsection{\setlength{\rightskip}{0pt plus 5cm}void CFile\-Event::Monitor\-Writable (bool {\em f\-Writable} = true)}\label{classCFileEvent_a13}


Select if the monitor should watch writability. This is delegated to the Monitor's Monitor\-Writable member function. \begin{Desc}
\item[Parameters: ]\par
\begin{description}
\item[{\em 
f\-Writable}]- true to watch writability false to disable \end{description}
\end{Desc}


Definition at line 626 of file CFile\-Event.cpp.

References CEvent::get\-Monitor().

Referenced by Setup\-Monitor().\index{CFileEvent@{CFile\-Event}!OnException@{OnException}}
\index{OnException@{OnException}!CFileEvent@{CFile\-Event}}
\subsubsection{\setlength{\rightskip}{0pt plus 5cm}void CFile\-Event::On\-Exception (iostream \& {\em r\-Stream})\hspace{0.3cm}{\tt  [virtual]}}\label{classCFileEvent_a17}


Called when the file descriptor has some exception condition.  Normally, if an application is interested in this, the programmer will create a  subclass of a CFile\-Event which implements this member function in a non null way. Default action is an implemented no-op. This requires the programmer to only provide implementations for the members s/he needs. The alternative (pure virtual member), requires a programmer to implement (even if only as a no-op), all members. \begin{Desc}
\item[Parameters: ]\par
\begin{description}
\item[{\em 
r\-Stream}]- Stream open on the file. \end{description}
\end{Desc}


Definition at line 683 of file CFile\-Event.cpp.

Referenced by CFile\-Event\-Reactor::On\-Exception().\index{CFileEvent@{CFile\-Event}!OnReadable@{OnReadable}}
\index{OnReadable@{OnReadable}!CFileEvent@{CFile\-Event}}
\subsubsection{\setlength{\rightskip}{0pt plus 5cm}void CFile\-Event::On\-Readable (istream \& {\em str})\hspace{0.3cm}{\tt  [virtual]}}\label{classCFileEvent_a15}


Called when a file descriptor becomes readable. Normally, if an  application is interested in this, the programmer will create a subclass of CFile\-Event which implements this member function in a non-null way. Default action is an implemented No-op. By implementing (rather than making this a pure virtual member), the programmer needs only to provide the members s/he's interested in.\begin{Desc}
\item[Parameters: ]\par
\begin{description}
\item[{\em 
r\-Stream}]- Stream open on the file. \end{description}
\end{Desc}


Reimplemented in {\bf CServer\-Connection\-Event} {\rm (p.\,\pageref{classCServerConnectionEvent_a13})}, and {\bf CServer\-Instance} {\rm (p.\,\pageref{classCServerInstance_a8})}.

Definition at line 654 of file CFile\-Event.cpp.

Referenced by CFile\-Event\-Reactor::On\-Readable().\index{CFileEvent@{CFile\-Event}!OnTimeout@{OnTimeout}}
\index{OnTimeout@{OnTimeout}!CFileEvent@{CFile\-Event}}
\subsubsection{\setlength{\rightskip}{0pt plus 5cm}void CFile\-Event::On\-Timeout (iostream \& {\em r\-Stream})\hspace{0.3cm}{\tt  [virtual]}}\label{classCFileEvent_a18}


Calle dwhen a wait times out, and the caller has idicated that they want to pay attention to timeouts. \begin{Desc}
\item[ams str - reference to stream open on file.]\par
\end{Desc}


Definition at line 692 of file CFile\-Event.cpp.

Referenced by CFile\-Event\-Reactor::On\-Timeout().\index{CFileEvent@{CFile\-Event}!OnWritable@{OnWritable}}
\index{OnWritable@{OnWritable}!CFileEvent@{CFile\-Event}}
\subsubsection{\setlength{\rightskip}{0pt plus 5cm}void CFile\-Event::On\-Writable (ostream \& {\em r\-Stream})\hspace{0.3cm}{\tt  [virtual]}}\label{classCFileEvent_a16}


Called when the file descriptor becomes writable. Normally, if an application is interested in this, the programmer will create a  subclass of a CFile\-Event which implements this member function in a non null way. Default action is an implemented no-op. This requires the programmer to only provide implementations for the members s/he needs. The alternative (pure virtual member), requires a programmer to implement (even if only as a no-op), all members. \begin{Desc}
\item[Parameters: ]\par
\begin{description}
\item[{\em 
r\-Stream}]- Stream open on the file. \end{description}
\end{Desc}


Definition at line 668 of file CFile\-Event.cpp.

Referenced by CFile\-Event\-Reactor::On\-Writable().\index{CFileEvent@{CFile\-Event}!operator=@{operator=}}
\index{operator=@{operator=}!CFileEvent@{CFile\-Event}}
\subsubsection{\setlength{\rightskip}{0pt plus 5cm}CFile\-Event\& CFile\-Event::operator= (const CFile\-Event \& {\em rhs})\hspace{0.3cm}{\tt  [private]}}\label{classCFileEvent_c1}


\index{CFileEvent@{CFile\-Event}!operator==@{operator==}}
\index{operator==@{operator==}!CFileEvent@{CFile\-Event}}
\subsubsection{\setlength{\rightskip}{0pt plus 5cm}int CFile\-Event::operator== (const CFile\-Event \& {\em r\-Event})\hspace{0.3cm}{\tt  [private]}}\label{classCFileEvent_c2}


\index{CFileEvent@{CFile\-Event}!SetupMonitor@{SetupMonitor}}
\index{SetupMonitor@{SetupMonitor}!CFileEvent@{CFile\-Event}}
\subsubsection{\setlength{\rightskip}{0pt plus 5cm}void CFile\-Event::Setup\-Monitor (int {\em Access\-Mask})\hspace{0.3cm}{\tt  [protected]}}\label{classCFileEvent_b1}


Set up the acessibility of the monitor. This means that the access mask is translated into an initial set of calls to  Monitor\-Readable, Monitor\-Writable.\begin{Desc}
\item[Parameters: ]\par
\begin{description}
\item[{\em 
Access\-Mask}]- Bitwise or of the access requested.\end{description}
\end{Desc}
Legitimate bits in Access\-Mask are:\begin{CompactItemize}
\item 
{\bf CFile\-Event::readable} {\rm (p.\,\pageref{classCFileEvent_p0})} - Monitor readability.\item 
CFile\-Event::writable - Monitor writability.\item 
{\bf CFile\-Event::exceptions} {\rm (p.\,\pageref{classCFileEvent_p2})} - Monitor for exceptions. \end{CompactItemize}


Definition at line 762 of file CFile\-Event.cpp.

References exceptions, Monitor\-Exceptions(), Monitor\-Readable(), Monitor\-Writable(), readable, and writeable.

Referenced by CFile\-Event().

\subsection{Member Data Documentation}
\index{CFileEvent@{CFile\-Event}!exceptions@{exceptions}}
\index{exceptions@{exceptions}!CFileEvent@{CFile\-Event}}
\subsubsection{\setlength{\rightskip}{0pt plus 5cm}int CFile\-Event::exceptions = 4\hspace{0.3cm}{\tt  [static]}}\label{classCFileEvent_p2}


Bitmask for exception monitoring.



Definition at line 304 of file CFile\-Event.cpp.

Referenced by Setup\-Monitor().\index{CFileEvent@{CFile\-Event}!m_fCloseOnDestroy@{m\_\-fCloseOnDestroy}}
\index{m_fCloseOnDestroy@{m\_\-fCloseOnDestroy}!CFileEvent@{CFile\-Event}}
\subsubsection{\setlength{\rightskip}{0pt plus 5cm}bool CFile\-Event::m\_\-f\-Close\-On\-Destroy\hspace{0.3cm}{\tt  [private]}}\label{classCFileEvent_o1}


true if we opened the file.



Definition at line 358 of file CFile\-Event.h.

Referenced by CFile\-Event(), Describe\-Self(), and get\-Close\-On\-Destory().\index{CFileEvent@{CFile\-Event}!m_nFd@{m\_\-nFd}}
\index{m_nFd@{m\_\-nFd}!CFileEvent@{CFile\-Event}}
\subsubsection{\setlength{\rightskip}{0pt plus 5cm}int CFile\-Event::m\_\-n\-Fd\hspace{0.3cm}{\tt  [private]}}\label{classCFileEvent_o0}


File descriptor.



Definition at line 357 of file CFile\-Event.h.

Referenced by CFile\-Event(), Create\-Monitor(), Describe\-Self(), get\-Fd(), and $\sim$CFile\-Event().\index{CFileEvent@{CFile\-Event}!readable@{readable}}
\index{readable@{readable}!CFileEvent@{CFile\-Event}}
\subsubsection{\setlength{\rightskip}{0pt plus 5cm}int CFile\-Event::readable = 1\hspace{0.3cm}{\tt  [static]}}\label{classCFileEvent_p0}


Bitmask for readable monitoring.



Definition at line 302 of file CFile\-Event.cpp.

Referenced by Setup\-Monitor().\index{CFileEvent@{CFile\-Event}!writeable@{writeable}}
\index{writeable@{writeable}!CFileEvent@{CFile\-Event}}
\subsubsection{\setlength{\rightskip}{0pt plus 5cm}int CFile\-Event::writeable = 2\hspace{0.3cm}{\tt  [static]}}\label{classCFileEvent_p1}


Bitmask for writable monitoring.



Definition at line 303 of file CFile\-Event.cpp.

Referenced by Setup\-Monitor().

The documentation for this class was generated from the following files:\begin{CompactItemize}
\item 
{\bf CFile\-Event.h}\item 
{\bf CFile\-Event.cpp}\end{CompactItemize}
