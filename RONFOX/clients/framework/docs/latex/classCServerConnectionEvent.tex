\section{CServer\-Connection\-Event  Class Reference}
\label{classCServerConnectionEvent}\index{CServerConnectionEvent@{CServer\-Connection\-Event}}
$\backslash$class: CServer\-Connection\-Event $\backslash$file: {\bf CServer\-Connection\-Event.h}. 


{\tt \#include $<$CServer\-Connection\-Event.h$>$}

Inheritance diagram for CServer\-Connection\-Event::\begin{figure}[H]
\begin{center}
\leavevmode
\includegraphics[height=5cm]{classCServerConnectionEvent}
\end{center}
\end{figure}
\subsection*{Public Methods}
\begin{CompactItemize}
\item 
{\bf CServer\-Connection\-Event} ({\bf CSocket} \&sock)
\item 
{\bf CServer\-Connection\-Event} (const char $\ast$p\-Name, {\bf CSocket} \&sock)
\item 
{\bf CServer\-Connection\-Event} (const string \&r\-Name, {\bf CSocket} \&sock)
\item 
{\bf CServer\-Connection\-Event} (const string \&r\-Service\-Name)
\item 
{\bf CServer\-Connection\-Event} (const char $\ast$p\-Name, const string \&r\-Service\-Name)
\item 
{\bf CServer\-Connection\-Event} (const string \&r\-Name, const string \&r\-Service\-Name)
\item 
{\bf CServer\-Connection\-Event} (int n\-Fd)
\item 
{\bf CServer\-Connection\-Event} (const char $\ast$p\-Name, int n\-Fd)
\item 
{\bf CServer\-Connection\-Event} (const string \&r\-Name, int n\-Fd)
\item 
{\bf $\sim$CServer\-Connection\-Event} ()
\item 
{\bf CSocket} $\ast$ {\bf get\-Socket} ()
\item 
bool {\bf is\-Dynamic\-Socket} () const
\item 
virtual void {\bf On\-Connection} ({\bf CSocket} $\ast$p\-Peer)
\item 
virtual void {\bf On\-Readable} (istream \&r\-Stream)
\item 
virtual string {\bf Describe\-Self} ()
\end{CompactItemize}
\subsection*{Protected Methods}
\begin{CompactItemize}
\item 
void {\bf set\-Socket} ({\bf CSocket} $\ast$p\-Socket)
\item 
void {\bf set\-Dynamic} (bool f\-Dyn)
\item 
void {\bf Configure\-Socket} (const string \&r\-Svc\-Name)
\item 
void {\bf Configure\-Socket} (int n\-Port)
\item 
int {\bf Protocol} ()
\end{CompactItemize}
\subsection*{Private Methods}
\begin{CompactItemize}
\item 
{\bf CServer\-Connection\-Event} (const CServer\-Connection\-Event \&rhs)
\item 
CServer\-Connection\-Event \& {\bf operator=} (const CServer\-Connection\-Event \&rhs)
\item 
int {\bf operator==} (const CServer\-Connection\-Event \&rhs)
\end{CompactItemize}
\subsection*{Private Attributes}
\begin{CompactItemize}
\item 
{\bf CSocket} $\ast$ {\bf m\_\-p\-Service}
\begin{CompactList}\small\item\em Server listener socket.\item\end{CompactList}\item 
bool {\bf m\_\-f\-Delete\-Socket}
\begin{CompactList}\small\item\em true if destructor delete socket.\item\end{CompactList}\end{CompactItemize}


\subsection{Detailed Description}
$\backslash$class: CServer\-Connection\-Event $\backslash$file: {\bf CServer\-Connection\-Event.h}.

This class encapsulates the process of listening for and accepting client connections to a Tcp/Ip service. The event is policy free in the sense that all it does is set up an encapsulated {\bf CSocket} {\rm (p.\,\pageref{classCSocket})} to listen on a specified port and listen for connections on the socket within a thread. When a connection request is received... it is accepted and the resulting socket is passed to {\bf On\-Connection}() {\rm (p.\,\pageref{classCServerConnectionEvent_a12})}

The user of this class in general will subclass the server and override the default (no-op) implementation of {\bf On\-Connection}() {\rm (p.\,\pageref{classCServerConnectionEvent_a12})} 



Definition at line 322 of file CServer\-Connection\-Event.h.

\subsection{Constructor \& Destructor Documentation}
\index{CServerConnectionEvent@{CServer\-Connection\-Event}!CServerConnectionEvent@{CServerConnectionEvent}}
\index{CServerConnectionEvent@{CServerConnectionEvent}!CServerConnectionEvent@{CServer\-Connection\-Event}}
\subsubsection{\setlength{\rightskip}{0pt plus 5cm}CServer\-Connection\-Event::CServer\-Connection\-Event ({\bf CSocket} \& {\em r\-Socket})}\label{classCServerConnectionEvent_a0}


Construct an anonymous connection listener from an existing socket object. In this case, it will not be necessary for the destructor to eliminate the socket.\begin{Desc}
\item[Parameters: ]\par
\begin{description}
\item[{\em 
sock}]- An existing socket, bound, listening and ready to accept when the fd becomes readable. \end{description}
\end{Desc}


Definition at line 305 of file CServer\-Connection\-Event.cpp.\index{CServerConnectionEvent@{CServer\-Connection\-Event}!CServerConnectionEvent@{CServerConnectionEvent}}
\index{CServerConnectionEvent@{CServerConnectionEvent}!CServerConnectionEvent@{CServer\-Connection\-Event}}
\subsubsection{\setlength{\rightskip}{0pt plus 5cm}CServer\-Connection\-Event::CServer\-Connection\-Event (const char $\ast$ {\em pname}, {\bf CSocket} \& {\em r\-Socket})}\label{classCServerConnectionEvent_a1}


Construct a named connection listener from and existing socket object.\begin{Desc}
\item[Parameters: ]\par
\begin{description}
\item[{\em 
p\-Name}]- char$\ast$ pointer to the name of the event. \item[{\em 
r\-Socket}]- Reference to existing socket. \end{description}
\end{Desc}


Definition at line 317 of file CServer\-Connection\-Event.cpp.\index{CServerConnectionEvent@{CServer\-Connection\-Event}!CServerConnectionEvent@{CServerConnectionEvent}}
\index{CServerConnectionEvent@{CServerConnectionEvent}!CServerConnectionEvent@{CServer\-Connection\-Event}}
\subsubsection{\setlength{\rightskip}{0pt plus 5cm}CServer\-Connection\-Event::CServer\-Connection\-Event (const string \& {\em r\-Name}, {\bf CSocket} \& {\em r\-Socket})}\label{classCServerConnectionEvent_a2}


Construct a named connnection listener from an existing socket object.\begin{Desc}
\item[Parameters: ]\par
\begin{description}
\item[{\em 
r\-Name}]- Reference to the object name desired. \item[{\em 
r\-Socket-}]Socket to create the listener around. The socket must be in the listening state, by the time the event is enabled/scheduled. \end{description}
\end{Desc}


Definition at line 331 of file CServer\-Connection\-Event.cpp.\index{CServerConnectionEvent@{CServer\-Connection\-Event}!CServerConnectionEvent@{CServerConnectionEvent}}
\index{CServerConnectionEvent@{CServerConnectionEvent}!CServerConnectionEvent@{CServer\-Connection\-Event}}
\subsubsection{\setlength{\rightskip}{0pt plus 5cm}CServer\-Connection\-Event::CServer\-Connection\-Event (const string \& {\em r\-Service\-Name})}\label{classCServerConnectionEvent_a3}


Construct an anonymous connection listener from a service name. Since we need to construct the file event with an fd, that will be done by an explicit socket(2) call. The m\_\-p\-Service will be dynamically allocated and configured from the fd, and the service information in the constructor body.\begin{Desc}
\item[Parameters: ]\par
\begin{description}
\item[{\em 
r\-Service\-Name}]- Name of the service to connect to. The {\bf CSocket} {\rm (p.\,\pageref{classCSocket})} type accepts either a name which translates via getservbyname, or a stringified service number. \end{description}
\end{Desc}


Definition at line 349 of file CServer\-Connection\-Event.cpp.

References Configure\-Socket().\index{CServerConnectionEvent@{CServer\-Connection\-Event}!CServerConnectionEvent@{CServerConnectionEvent}}
\index{CServerConnectionEvent@{CServerConnectionEvent}!CServerConnectionEvent@{CServer\-Connection\-Event}}
\subsubsection{\setlength{\rightskip}{0pt plus 5cm}CServer\-Connection\-Event::CServer\-Connection\-Event (const char $\ast$ {\em p\-Name}, const string \& {\em r\-Service\-Name})}\label{classCServerConnectionEvent_a4}


Construct a named connection listener from a service name. See above for more information. \begin{Desc}
\item[Parameters: ]\par
\begin{description}
\item[{\em 
p\-Name}]- char$\ast$ name to be given to the connection listener. \item[{\em 
r\-Service\-Name}]- Name of the service to connect to. The {\bf CSocket} {\rm (p.\,\pageref{classCSocket})} type accepts either a name which translates via getservbyname, or a stringified service number. \end{description}
\end{Desc}


Definition at line 365 of file CServer\-Connection\-Event.cpp.

References Configure\-Socket().\index{CServerConnectionEvent@{CServer\-Connection\-Event}!CServerConnectionEvent@{CServerConnectionEvent}}
\index{CServerConnectionEvent@{CServerConnectionEvent}!CServerConnectionEvent@{CServer\-Connection\-Event}}
\subsubsection{\setlength{\rightskip}{0pt plus 5cm}CServer\-Connection\-Event::CServer\-Connection\-Event (const string \& {\em r\-Name}, const string \& {\em r\-Service\-Name})}\label{classCServerConnectionEvent_a5}




Definition at line 373 of file CServer\-Connection\-Event.cpp.

References Configure\-Socket().\index{CServerConnectionEvent@{CServer\-Connection\-Event}!CServerConnectionEvent@{CServerConnectionEvent}}
\index{CServerConnectionEvent@{CServerConnectionEvent}!CServerConnectionEvent@{CServer\-Connection\-Event}}
\subsubsection{\setlength{\rightskip}{0pt plus 5cm}CServer\-Connection\-Event::CServer\-Connection\-Event (int {\em n\-Fd})}\label{classCServerConnectionEvent_a6}


Construct an anonymous Connection listener given a file id which represents a bound and listening socket.\begin{Desc}
\item[Parameters: ]\par
\begin{description}
\item[{\em 
n\-Fd}]- A file descriptor which is a bound and listening socket. \end{description}
\end{Desc}


Definition at line 388 of file CServer\-Connection\-Event.cpp.\index{CServerConnectionEvent@{CServer\-Connection\-Event}!CServerConnectionEvent@{CServerConnectionEvent}}
\index{CServerConnectionEvent@{CServerConnectionEvent}!CServerConnectionEvent@{CServer\-Connection\-Event}}
\subsubsection{\setlength{\rightskip}{0pt plus 5cm}CServer\-Connection\-Event::CServer\-Connection\-Event (const char $\ast$ {\em p\-Name}, int {\em n\-Fd})}\label{classCServerConnectionEvent_a7}


Construct a named connection listener given a file id which represents a  bound and listening socket.\begin{Desc}
\item[Parameters: ]\par
\begin{description}
\item[{\em 
p\-Name}]- char$\ast$ name to be given to the socket. \item[{\em 
n\-Fd}]- File descriptor representing a bound and listening sock. \end{description}
\end{Desc}


Definition at line 401 of file CServer\-Connection\-Event.cpp.\index{CServerConnectionEvent@{CServer\-Connection\-Event}!CServerConnectionEvent@{CServerConnectionEvent}}
\index{CServerConnectionEvent@{CServerConnectionEvent}!CServerConnectionEvent@{CServer\-Connection\-Event}}
\subsubsection{\setlength{\rightskip}{0pt plus 5cm}CServer\-Connection\-Event::CServer\-Connection\-Event (const string \& {\em r\-Name}, int {\em n\-Fd})}\label{classCServerConnectionEvent_a8}


Construct a named connection listener given a file id which represents a bound and listening socket.\begin{Desc}
\item[Parameters: ]\par
\begin{description}
\item[{\em 
r\-Name}]- string\& name to be given to the socket. \item[{\em 
n\-Fd}]- File descriptor representing a bound and listening socket. \end{description}
\end{Desc}


Definition at line 414 of file CServer\-Connection\-Event.cpp.\index{CServerConnectionEvent@{CServer\-Connection\-Event}!~CServerConnectionEvent@{$\sim$CServerConnectionEvent}}
\index{~CServerConnectionEvent@{$\sim$CServerConnectionEvent}!CServerConnectionEvent@{CServer\-Connection\-Event}}
\subsubsection{\setlength{\rightskip}{0pt plus 5cm}CServer\-Connection\-Event::$\sim$CServer\-Connection\-Event ()}\label{classCServerConnectionEvent_a9}


Destructor... if m\_\-f\-Delete socket is true, then the socket is deleted. otherwise just closed. 

Definition at line 425 of file CServer\-Connection\-Event.cpp.

References m\_\-p\-Service, and CSocket::Shutdown().\index{CServerConnectionEvent@{CServer\-Connection\-Event}!CServerConnectionEvent@{CServerConnectionEvent}}
\index{CServerConnectionEvent@{CServerConnectionEvent}!CServerConnectionEvent@{CServer\-Connection\-Event}}
\subsubsection{\setlength{\rightskip}{0pt plus 5cm}CServer\-Connection\-Event::CServer\-Connection\-Event (const CServer\-Connection\-Event \& {\em rhs})\hspace{0.3cm}{\tt  [private]}}\label{classCServerConnectionEvent_c0}




\subsection{Member Function Documentation}
\index{CServerConnectionEvent@{CServer\-Connection\-Event}!ConfigureSocket@{ConfigureSocket}}
\index{ConfigureSocket@{ConfigureSocket}!CServerConnectionEvent@{CServer\-Connection\-Event}}
\subsubsection{\setlength{\rightskip}{0pt plus 5cm}void CServer\-Connection\-Event::Configure\-Socket (int {\em n\-Port})\hspace{0.3cm}{\tt  [protected]}}\label{classCServerConnectionEvent_b3}


Utility function to configure a socket. The port id is  used to bind the socket and set it into listen mode.

\begin{Desc}
\item[Note: ]\par
It's possible that a socket can come into this function in the wrong mode if there are program errors. In that case, the {\bf CSocket} {\rm (p.\,\pageref{classCSocket})} calls made may throw an exception. We catch all exceptions, if necessary delete the socket object and rethrow to the caller.\end{Desc}
\begin{Desc}
\item[Parameters: ]\par
\begin{description}
\item[{\em 
n\-Port}]- The number of the port to bind the socket to.\end{description}
\end{Desc}
\begin{Desc}
\item[Note: ]\par
On non exceptional exit from this function, m\_\-p\-Service is bound to the selected service and the socket is in the listen state. \end{Desc}


Definition at line 513 of file CServer\-Connection\-Event.cpp.

References Configure\-Socket().\index{CServerConnectionEvent@{CServer\-Connection\-Event}!ConfigureSocket@{ConfigureSocket}}
\index{ConfigureSocket@{ConfigureSocket}!CServerConnectionEvent@{CServer\-Connection\-Event}}
\subsubsection{\setlength{\rightskip}{0pt plus 5cm}void CServer\-Connection\-Event::Configure\-Socket (const string \& {\em r\-Svc\-Name})\hspace{0.3cm}{\tt  [protected]}}\label{classCServerConnectionEvent_b2}


Utility function to configure a socket. The service name is  used to bind the socket and set it into listen mode.

\begin{Desc}
\item[Note: ]\par
It's possible that a socket can come into this function in the wrong mode if there are program errors. In that case, the {\bf CSocket} {\rm (p.\,\pageref{classCSocket})} calls made may throw an exception. We catch all exceptions, if necessary delete the socket object and rethrow to the caller.\end{Desc}
\begin{Desc}
\item[Parameters: ]\par
\begin{description}
\item[{\em 
r\-Svc\-Name}]- A service name which can either be an entry in the local system's service data base, or alternatively a numerically encoded port id.\end{description}
\end{Desc}
\begin{Desc}
\item[Note: ]\par
On non exceptional exit from this function, m\_\-p\-Service is bound to the selected service and the socket is in the listen state. \end{Desc}


Definition at line 485 of file CServer\-Connection\-Event.cpp.

References CSocket::Bind(), CSocket::Listen(), m\_\-f\-Delete\-Socket, and m\_\-p\-Service.

Referenced by Configure\-Socket(), and CServer\-Connection\-Event().\index{CServerConnectionEvent@{CServer\-Connection\-Event}!DescribeSelf@{DescribeSelf}}
\index{DescribeSelf@{DescribeSelf}!CServerConnectionEvent@{CServer\-Connection\-Event}}
\subsubsection{\setlength{\rightskip}{0pt plus 5cm}string CServer\-Connection\-Event::Describe\-Self ()\hspace{0.3cm}{\tt  [virtual]}}\label{classCServerConnectionEvent_a14}


Returns a string which describes this object. 

Reimplemented from {\bf CFile\-Event} {\rm (p.\,\pageref{classCFileEvent_a19})}.

Definition at line 552 of file CServer\-Connection\-Event.cpp.

References CEvent::Describe\-Self(), CSocket::get\-Socket\-Fd(), CSocket::get\-State(), m\_\-f\-Delete\-Socket, m\_\-p\-Service, and CSocket::State\-Name().\index{CServerConnectionEvent@{CServer\-Connection\-Event}!getSocket@{getSocket}}
\index{getSocket@{getSocket}!CServerConnectionEvent@{CServer\-Connection\-Event}}
\subsubsection{\setlength{\rightskip}{0pt plus 5cm}{\bf CSocket}$\ast$ CServer\-Connection\-Event::get\-Socket ()\hspace{0.3cm}{\tt  [inline]}}\label{classCServerConnectionEvent_a10}




Definition at line 363 of file CServer\-Connection\-Event.h.\index{CServerConnectionEvent@{CServer\-Connection\-Event}!isDynamicSocket@{isDynamicSocket}}
\index{isDynamicSocket@{isDynamicSocket}!CServerConnectionEvent@{CServer\-Connection\-Event}}
\subsubsection{\setlength{\rightskip}{0pt plus 5cm}bool CServer\-Connection\-Event::is\-Dynamic\-Socket () const\hspace{0.3cm}{\tt  [inline]}}\label{classCServerConnectionEvent_a11}




Definition at line 367 of file CServer\-Connection\-Event.h.

References m\_\-f\-Delete\-Socket.\index{CServerConnectionEvent@{CServer\-Connection\-Event}!OnConnection@{OnConnection}}
\index{OnConnection@{OnConnection}!CServerConnectionEvent@{CServer\-Connection\-Event}}
\subsubsection{\setlength{\rightskip}{0pt plus 5cm}void CServer\-Connection\-Event::On\-Connection ({\bf CSocket} $\ast$ {\em p\-Peer})\hspace{0.3cm}{\tt  [virtual]}}\label{classCServerConnectionEvent_a12}


This member function is called whenever a connection is available. The default implementation rejects the connection and destroys the connection object. It's assumed that the user of this class will subclass it and override this member with code to generate an object to manage the connection. How this is done depends on the needs of the application. One typical way to do this would be to create a {\bf CServer\-Instance} {\rm (p.\,\pageref{classCServerInstance})} derived object and enable it. This would start up a server instance in a separate thread, to handle requests in a application synchronized manner.\begin{Desc}
\item[Parameters: ]\par
\begin{description}
\item[{\em 
p\-Peer}]- pointer to dynamically created socket representing the connection peer. \end{description}
\end{Desc}


Definition at line 445 of file CServer\-Connection\-Event.cpp.

References CSocket::Shutdown().

Referenced by On\-Readable().\index{CServerConnectionEvent@{CServer\-Connection\-Event}!OnReadable@{OnReadable}}
\index{OnReadable@{OnReadable}!CServerConnectionEvent@{CServer\-Connection\-Event}}
\subsubsection{\setlength{\rightskip}{0pt plus 5cm}void CServer\-Connection\-Event::On\-Readable (istream \& {\em r\-Stream})\hspace{0.3cm}{\tt  [virtual]}}\label{classCServerConnectionEvent_a13}


Called when the service socket becomes readable. This indicates that a connection request is available. The request is serviced by an accept on the service socket and On\-Connection is called to handle the actual service.  \begin{Desc}
\item[Note: ]\par
On\-Readable is called while holding the application serialization mutex. \par
On\-Readable in general need not be overridden in subclasses.\end{Desc}
\begin{Desc}
\item[Parameters: ]\par
\begin{description}
\item[{\em 
r\-Stream}]- istream on the socket... ignored. \end{description}
\end{Desc}


Reimplemented from {\bf CFile\-Event} {\rm (p.\,\pageref{classCFileEvent_a15})}.

Definition at line 462 of file CServer\-Connection\-Event.cpp.

References CSocket::Accept(), m\_\-p\-Service, and On\-Connection().\index{CServerConnectionEvent@{CServer\-Connection\-Event}!operator=@{operator=}}
\index{operator=@{operator=}!CServerConnectionEvent@{CServer\-Connection\-Event}}
\subsubsection{\setlength{\rightskip}{0pt plus 5cm}CServer\-Connection\-Event\& CServer\-Connection\-Event::operator= (const CServer\-Connection\-Event \& {\em rhs})\hspace{0.3cm}{\tt  [private]}}\label{classCServerConnectionEvent_c1}


\index{CServerConnectionEvent@{CServer\-Connection\-Event}!operator==@{operator==}}
\index{operator==@{operator==}!CServerConnectionEvent@{CServer\-Connection\-Event}}
\subsubsection{\setlength{\rightskip}{0pt plus 5cm}int CServer\-Connection\-Event::operator== (const CServer\-Connection\-Event \& {\em rhs})\hspace{0.3cm}{\tt  [private]}}\label{classCServerConnectionEvent_c2}


\index{CServerConnectionEvent@{CServer\-Connection\-Event}!Protocol@{Protocol}}
\index{Protocol@{Protocol}!CServerConnectionEvent@{CServer\-Connection\-Event}}
\subsubsection{\setlength{\rightskip}{0pt plus 5cm}int CServer\-Connection\-Event::Protocol ()\hspace{0.3cm}{\tt  [protected]}}\label{classCServerConnectionEvent_b4}


Determine the correct protocol id for a socket(2) call. We only support tcp as the protocol. This function encapsulates the call to getprotobyname(3), and the decode of the arguments. getprotobyname(3) is assumed to be non-reentrant/non-recursive and therefore the call to it is serialized to the application.

Failures are not considered possible, so are asserted against, since the only acceptable protocol is \char`\"{}TCP\char`\"{}.\begin{Desc}
\item[Return values: ]\par
\begin{description}
\item[{\em 
int}]- a protocol id suitable for the last parameter of the socket(2) call. \end{description}
\end{Desc}


Definition at line 535 of file CServer\-Connection\-Event.cpp.

References CApplication\-Serializer::get\-Instance(), CThread\-Recursive\-Mutex::Lock(), and CThread\-Recursive\-Mutex::Un\-Lock().\index{CServerConnectionEvent@{CServer\-Connection\-Event}!setDynamic@{setDynamic}}
\index{setDynamic@{setDynamic}!CServerConnectionEvent@{CServer\-Connection\-Event}}
\subsubsection{\setlength{\rightskip}{0pt plus 5cm}void CServer\-Connection\-Event::set\-Dynamic (bool {\em f\-Dyn})\hspace{0.3cm}{\tt  [inline, protected]}}\label{classCServerConnectionEvent_b1}




Definition at line 376 of file CServer\-Connection\-Event.h.

References m\_\-f\-Delete\-Socket.\index{CServerConnectionEvent@{CServer\-Connection\-Event}!setSocket@{setSocket}}
\index{setSocket@{setSocket}!CServerConnectionEvent@{CServer\-Connection\-Event}}
\subsubsection{\setlength{\rightskip}{0pt plus 5cm}void CServer\-Connection\-Event::set\-Socket ({\bf CSocket} $\ast$ {\em p\-Socket})\hspace{0.3cm}{\tt  [inline, protected]}}\label{classCServerConnectionEvent_b0}




Definition at line 373 of file CServer\-Connection\-Event.h.

\subsection{Member Data Documentation}
\index{CServerConnectionEvent@{CServer\-Connection\-Event}!m_fDeleteSocket@{m\_\-fDeleteSocket}}
\index{m_fDeleteSocket@{m\_\-fDeleteSocket}!CServerConnectionEvent@{CServer\-Connection\-Event}}
\subsubsection{\setlength{\rightskip}{0pt plus 5cm}bool CServer\-Connection\-Event::m\_\-f\-Delete\-Socket\hspace{0.3cm}{\tt  [private]}}\label{classCServerConnectionEvent_o1}


true if destructor delete socket.



Definition at line 327 of file CServer\-Connection\-Event.h.

Referenced by Configure\-Socket(), Describe\-Self(), is\-Dynamic\-Socket(), and set\-Dynamic().\index{CServerConnectionEvent@{CServer\-Connection\-Event}!m_pService@{m\_\-pService}}
\index{m_pService@{m\_\-pService}!CServerConnectionEvent@{CServer\-Connection\-Event}}
\subsubsection{\setlength{\rightskip}{0pt plus 5cm}{\bf CSocket}$\ast$ CServer\-Connection\-Event::m\_\-p\-Service\hspace{0.3cm}{\tt  [private]}}\label{classCServerConnectionEvent_o0}


Server listener socket.



Definition at line 326 of file CServer\-Connection\-Event.h.

Referenced by Configure\-Socket(), Describe\-Self(), On\-Readable(), and $\sim$CServer\-Connection\-Event().

The documentation for this class was generated from the following files:\begin{CompactItemize}
\item 
{\bf CServer\-Connection\-Event.h}\item 
{\bf CServer\-Connection\-Event.cpp}\end{CompactItemize}
