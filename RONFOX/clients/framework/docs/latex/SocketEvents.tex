\section{Server Instance events.}\label{SocketEvents}


The class CServer\-Instance\-Event is used to encapsulate a simple model of client server interaction. The model implemented is that following connection:\begin{CompactItemize}
\item 
Clients make requests of the server by sending them a message.\item 
Servers, perform the request and optionally send a reply message to the client.\item 
Servers never initiate communication with the client without being prompted first by the client with a request message.\end{CompactItemize}
This simple model is sufficient to encompass a wide variety of application level protocols. Under this model, you can think of a CServer\-Instance\-Event as a {\bf CFile\-Event} {\rm (p.\,\pageref{classCFileEvent})} which is only sensitized to readability, and which operates on a  socket rather than a file descriptor.

In general, a CServer\-Instance\-Event is created by the  {\bf On\-Connection member function of a CServer\-Connectionvent derived object.} {\rm (p.\,\pageref{classCServerConnectionEvent_a12})} For more information, see {\bf Server accept events.} {\rm (p.\,\pageref{ServerEvent})}

The sample application below implements an echo server. The class to examine closely is the Echo\-Server class, its implementation and how it is constructed from the connection socket by the listener object (of type Echo\-Listener). Note as well how the Reaper helper class and object is used to delete the resources associated with Echo\-Server objects which are no longer active.



\footnotesize\begin{verbatim}#include <spectrodaq.h>
#include <SpectroFramework.h>
#include <list>
#include <string>


typedef list<CEvent*> EventList;

// Reaper objects are timed events which delete dead process objects:
//
class Reaper : public CTimerEvent
{
private:
  EventList m_DeletePending;
public:
  Reaper(const char* pName);

  void QueueEvent(CEvent* pEvent);
  CEvent* DeQueueEvent();

  virtual void OnTimer();
};
// Implementation of Reaper:

Reaper::Reaper(const char* pname) :
  CTimerEvent(pname, 1000, true) {}

void 
Reaper::QueueEvent(CEvent* pEvent)
{
  CApplicationSerializer::getInstance()->Lock(); // Don't assume this is done
  m_DeletePending.push_back(pEvent);             // in an event context. 
  CApplicationSerializer::getInstance()->UnLock();
}
CEvent* 
Reaper::DeQueueEvent()          // Returns NULL if empty queue or front not yet
{                               // inactive... assumed to run locked.
  if(m_DeletePending.empty()) return (CEvent*)NULL;

  CEvent* pItem = m_DeletePending.front();
  if(pItem->isActive()) {
    m_DeletePending.pop_front();
    return pItem;
  }
  else {
    return (CEvent*)NULL;
  }
}
void 
Reaper::OnTimer()
{
  CEvent* pEvent;
  while(pEvent = DeQueueEvent()) {
    delete pEvent;
  }
}

// Server instance. Echoes client requests on client channel until
// client exits.. at exit time, disables self and enters the object
// on the delete pending queue of a reaper.

class EchoServer : public CServerInstance
{
  Reaper& m_GrimReaper;
public:
  EchoServer(CSocket* pSocket, Reaper& pReapme);
  void OnRequest(CSocket* pSocket);
};

EchoServer::EchoServer(CSocket* pSocket, Reaper& rReapme) :
  CServerInstance(pSocket),
  m_GrimReaper(rReapme) {}

void
EchoServer::OnRequest(CSocket* pSocket) {
  char buffer[1024];
  int nread = pSocket->Read(buffer, sizeof(buffer)-1);
  if(nread <= 0) {              // Client exited or other error...
    Shutdown();                 // Shutdown our part of the connection.
    Disable();                  // Schedule thread exit and
    m_GrimReaper.QueueEvent(this); // Object deletion.
  } else {                      // Data available.
    pSocket->Write(buffer, nread);
  }
}


// Server listener.  Only new functionality is the OnConnection
// which creates a new server instance thread.

class EchoListener : public CServerConnectionEvent
{
  Reaper& m_GrimReaper;
public:
  EchoListener(const char* pName, const string& rservice, Reaper& rReaper);
  virtual void OnConnection(CSocket* pSocket);
};

EchoListener::EchoListener(const char* pName, const string& rservice,
                           Reaper& rReaper) :
  CServerConnectionEvent(pName, rservice),
  m_GrimReaper(rReaper)
{}
void
EchoListener::OnConnection(CSocket* pSocket)
{
  EchoServer* pServer = new EchoServer(pSocket, m_GrimReaper);
  pServer->Enable();
}


class MyApp : public DAQROCNode
{
protected:
  int operator()(int argc, char** pargv);

};
int
MyApp::operator()(int argc, char** pargv)
{
  Reaper theReaper("GrimReaper");
  theReaper.Enable();           // Start off the grim reaper.

  EchoListener Listen("EchoListen", string("2048"), theReaper);
  Listen.Enable();

  DAQThreadId id = Listen.getThreadId();
  Join(id);
  
};

MyApp theApplication;
\end{verbatim}\normalsize 


