\section{CXt\-Event\-Loop.cpp File Reference}
\label{CXtEventLoop_8cpp}\index{CXtEventLoop.cpp@{CXt\-Event\-Loop.cpp}}
{\tt \#include \char`\"{}CXt\-Event\-Loop.h\char`\"{}}\par
{\tt \#include \char`\"{}CApplication\-Serializer.h\char`\"{}}\par
\subsection*{Variables}
\begin{CompactItemize}
\item 
const char $\ast$ {\bf Copyright} = \char`\"{}(C) Copyright Michigan State University 2002, All rights reserved\char`\"{}
\end{CompactItemize}


\subsection{Detailed Description}


$\backslash$class {\bf CXt\-Event\-Loop} {\rm (p.\,\pageref{classCXtEventLoop})}  Encapsulates an occurance of an Xt event loop.  The main loop synchronizes the event loop thread with the application each pass through the  xt event loop e.g. the event loop looks like:

while(1) \{ Xt\-Get\-Event() Lock\-Mutex() Xt\-Dispatch\-Event(); Unlock\-Mutex(); yield(); // Let someone else run. \}

This implies that work procedures and timer procs are also synchonrized to the application. Note that this synchronization can be costly if there are work procedures continuously active.



Definition in file {\bf CXt\-Event\-Loop.cpp}.

\subsection{Variable Documentation}
\index{CXtEventLoop.cpp@{CXt\-Event\-Loop.cpp}!Copyright@{Copyright}}
\index{Copyright@{Copyright}!CXtEventLoop.cpp@{CXt\-Event\-Loop.cpp}}
\subsubsection{\setlength{\rightskip}{0pt plus 5cm}const char$\ast$ Copyright = \char`\"{}(C) Copyright Michigan State University 2002, All rights reserved\char`\"{}\hspace{0.3cm}{\tt  [static]}}\label{CXtEventLoop_8cpp_a0}




Definition at line 278 of file CXt\-Event\-Loop.cpp.